\documentclass{article}
\usepackage[utf8] {inputenc}
\usepackage [left=6cm, right=5cm, top=6cm, bottom=5cm]{geometry}

\parindent=0cm

\begin{document}
\begin{center}
\huge  \textbf{Commandes utilisateur LS(1) LS(1)} \\
\normalsize \textit {Arnaud Paradeis - Nael Boussetta - Théo Bolmont}\\[4cm]
\end{center}










\par NOM\\
		\textit{\textbf{ls - Liste du contenu des répertoires.}}\\
\par SYNOPSIS\\
		\textit{\textbf{ls [OPTION]... [FICHIER]...}}\\
\par DESCRIPTION\\
		\textit{\textbf{Liste les informations sur les FICHIERS (le répertoire courant par défaut).
       Les entrées sont triées par ordre alphabétique si aucune des options -cftuvSUX ou --sort n'est spécifiée.
       spécifié. Les arguments obligatoires des options longues sont également obligatoires pour les options courtes.
       aussi.}}\\

\par --a, --all\\
		\textit{\textbf{n'ignore pas les entrées commençant par .}}\\

\par -A, --almost-all\\
		\textit{\textbf{n'énumère pas les sous-entendus . et .}}\\

\par --author\\
              \textit{\textbf{avec -l, affiche l'auteur de chaque fichier}}\\

   \par-b, --escape\\
              \textit{\textbf{affiche les échappements de style C pour les caractères non graphiques}}\\

       \par--block-size=SIZE\\
             \textit{\textbf{ avec -l, met les tailles à l'échelle par SIZE lors de l'impression ; par exemple,
              '--block-size=M' ; voir le format SIZE ci-dessous}}\\

      \par -B, --ignore-backups\\
               \textit{\textbf{ne répertorie pas les entrées implicites se terminant par ~}}\\

      \par -c avec -lt \\
		  \textit{\textbf{trier par, et afficher, ctime (heure de la dernière modification de
              informations sur l'état du fichier) ; avec -l : afficher ctime et trier par nom ;
              sinon : trier par ctime, le plus récent en premier.}}\\

       \par-C \\
	 \textit{\textbf{liste les entrées par colonnes}}\\


       \par --color[=WHEN] \\
               \textit{\textbf{colore la sortie ; WHEN peut être 'always' (par défaut si omis),
              auto', ou 'jamais' ; plus d'informations ci-dessous}}\\

       \par-d, --directory\\
              \textit{\textbf{liste les répertoires eux-mêmes, pas leur contenu}}\\

      \par -D, --dired\\
              \textit{\textbf{génère une sortie conçue pour le mode dired d'Emacs}}\\

       \par-f \\
		 \textit{\textbf{ne pas trier, activer -aU, désactiver -ls --color}}\\

       \par -F, --classify\\
               \textit{\textbf{ajouter un indicateur (l'un des */=>@|) aux entrées}}\\

      \par --file-type\\
              \textit{\textbf{de même, sauf qu'il ne faut pas ajouter '*'.}}\\

       \par --format=WORD\\
              \textit{\textbf{across -x, commas -m, horizontal -x, long -l, single-column -1,
              verbeux -l, vertical -C}}\\

       \par --full-time\\
              \textit{\textbf{comme -l --time-style=full-iso}}\\

       \par-g comme -l\\
	 \textit{\textbf{mais sans liste de propriétaires}}\\

       \par --group-directories-first\\
              \textit{\textbf{regroupe les répertoires avant les fichiers ;
		peut être complété par une option --sort, mais toute utilisation de
              --sort=none (-U) désactive le regroupement.}}\\

       \par -G, --no-group\\
               \textit{\textbf{dans une longue liste, ne pas imprimer les noms de groupe}}\\

       \par-h, --human-readable\\
               \textit{\textbf{avec -l et -s, imprime les tailles comme 1K 234M 2G etc.}}\\

       \par--si \\
		 \textit{\textbf{de même, mais utiliser des puissances de 1000 et non 1024}}\\

        \par-H, --dereference-command-line\\
              \textit{\textbf{suit les liens symboliques listés sur la ligne de commande}}\\

      \par --dereference-command-line-symlink-to-dir\\
              \textit{\textbf{suit chaque lien symbolique de la ligne de commande}}\\

              
       \par--hide=PATTERN\\
              \textit{\textbf{ne pas lister les entrées implicites correspondant au PATTERN de l'interpréteur de commandes (remplacé
              par -a ou -A)}}\\
\par--hyperlien[=WHEN]\\
              \textit{\textbf{Hyperlien vers les noms de fichiers ; WHEN peut être 'always' (par défaut si omis),
              auto', ou 'jamais'.}}\\

       \par--indicator-style=WORD\\
               \textit{\textbf{ajoute un indicateur de style WORD aux noms des entrées : aucun (par défaut),
              slash (-p), file-type (--file-type), classify (-F)}}\\
   
       \par -i, --inode\\
              \textit{\textbf{ affiche le numéro d'index de chaque fichier}}\\

        \par-I, --ignore=PATTERN\\
               \textit{\textbf{ne pas lister les entrées implicites correspondant au PATTERN du shell}}\\

       \par-k, --kibibytes\\
              \textit{\textbf{ utilise par défaut des blocs de 1024 octets pour l'utilisation du disque ; utilisé seulement avec -s
              et les totaux par répertoire}}\\

      \par -l \\   \textit{\textbf{ utilise un format de liste long}}\\

      \par -L, --dereference\\  
             \textit{\textbf{ lors de l'affichage des informations sur les fichiers pour un lien symbolique, affiche les informations sur le fichier auquel le lien fait référence.
              pour le fichier auquel le lien fait référence plutôt que pour le lien lui-même.
              lui-même}}\\

       \par-m\\   \textit{\textbf{remplir la largeur avec une liste d'entrées séparées par des virgules}}\\

       \par -n, --numeric-uid-gid\\  
               \textit{\textbf{comme -l, mais liste les ID numériques d'utilisateurs et de groupes}}\\

        \par-N, --literal\\  
              \textit{\textbf{affiche les noms des entrées sans les citer}}\\

       \par -o \\      \textit{\textbf{comme -l, mais ne liste pas les informations de groupe}}\\

       \par -p, --indicator-style=slash \\ 
              \textit{\textbf{ajoute l'indicateur / aux répertoires}}\\

        \par-q, --hide-control-chars\\
              \textit{\textbf{affiche ? à la place des caractères non graphiques}}\\

        \par-Q, --quote-name\\
              \textit{\textbf{met les noms des entrées entre guillemets doubles}}\\


       \par-r, --reverse\\
               \textit{\textbf{inverser l'ordre pendant le tri}}\\

      \par -R, --recursive\\
              \textit{\textbf{liste les sous-répertoires récursivement}}\\

      \par -s, --size\\
             \textit{\textbf{  affiche la taille allouée de chaque fichier, en blocs}}\\

       \par-S\\          \textit{\textbf{ trier par taille de fichier, le plus grand en premier}}\\

       \par--sort=WORD\\ 
              \textit{\textbf{trie par MOT au lieu du nom : aucun (-U), taille (-S), temps (-t),
              version (-v), extension (-X)}}\\

       \par --time=MOT\\ 
              \textit{\textbf{avec -l, affiche l'heure sous forme de MOT au lieu de l'heure de modification par défaut :
              atime ou accès ou utilisation (-u) ; ctime ou état (-c) ; utilise également
              l'heure spécifiée comme clé de tri si --sort=time (le plus récent en premier)}}\\

   

        \par -t \\      \textit{\textbf{trier par heure de modification, la plus récente en premier}}\\

      \par -T, --tabsize=COLS\\  
              \textit{\textbf{suppose que la tabulation s'arrête à chaque COLS au lieu de 8}}\\

         \par-u avec -lt :\\    \textit{\textbf{trier et afficher l'heure d'accès ; avec -l : afficher l'heure d'accès et trier par nom ; sinon : trier par nom.
              d'accès et trier par nom ; sinon : trier par temps d'accès, le plus récent en premier
              d'abord}}\\

      \par -U \\
		 \textit{\textbf{ne trie pas ; liste les entrées dans l'ordre du répertoire.}}\\

       \par-v\\
		 \textit{\textbf{ tri naturel des numéros (de version) dans le texte}}\\

       \par-w, --width=COLS\\
              \textit{\textbf{définit la largeur de sortie à COLS.  0 signifie aucune limite}}\\

      \par -x\\
		\textit{\textbf{ liste les entrées par lignes plutôt que par colonnes}}\\

      \par -X\\
	 \textit{\textbf{trier alphabétiquement par extension d'entrée}}\\

       \par-Z, --context\\
              \textit{\textbf{ affiche le contexte de sécurité de chaque fichier}}\\

        \par-1 \\
		\textit{\textbf{liste un fichier par ligne.  Évitez  avec -q ou -b.}}\\

       \par--help affiche cette aide et quitte\\

        \par--version\\


\end{document}