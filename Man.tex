\documentclass{article}
\usepackage[utf8] {inputenc}
\usepackage [left=6cm, right=5cm, top=6cm, bottom=5cm]{geometry}

\parindent=0cm

\begin{document}
\begin{center}
\huge  \textbf{Commandes utilisateur LS(1) LS(1)} \\
\normalsize \textit {Arnaud Paradeis - Nael Boussetta - Théo Bolmont}\\[4cm]
\end{center}










\par NOM\\
		\textit{\textbf{ls - Liste du contenu des répertoires.}}\\
\par SYNOPSIS\\
		\textit{\textbf{ls [OPTION]... [FICHIER]...}}\\
\par DESCRIPTION\\
		\textit{\textbf{Liste les informations sur les FICHIERS (le répertoire courant par défaut).
       Les entrées sont triées par ordre alphabétique si aucune des options -cftuvSUX ou --sort n'est spécifiée.
       spécifié. Les arguments obligatoires des options longues sont également obligatoires pour les options courtes.
       aussi.}}\\

\par --a, --all\\
		\textit{\textbf{n'ignore pas les entrées commençant par .}}\\

\par -A, --almost-all\\
		\textit{\textbf{n'énumère pas les sous-entendus . et .}}\\

\par --author\\
              \textit{\textbf{avec -l, affiche l'auteur de chaque fichier}}\\

   \par-b, --escape\\
              \textit{\textbf{affiche les échappements de style C pour les caractères non graphiques}}\\

       \par--block-size=SIZE\\
             \textit{\textbf{ avec -l, met les tailles à l'échelle par SIZE lors de l'impression ; par exemple,
              '--block-size=M' ; voir le format SIZE ci-dessous}}\\

      \par -B, --ignore-backups\\
               \textit{\textbf{ne répertorie pas les entrées implicites se terminant par ~}}\\

      \par -c avec -lt \\
		  \textit{\textbf{trier par, et afficher, ctime (heure de la dernière modification de
              informations sur l'état du fichier) ; avec -l : afficher ctime et trier par nom ;
              sinon : trier par ctime, le plus récent en premier.}}\\

       \par-C \\
	 \textit{\textbf{liste les entrées par colonnes}}\\


       \par --color[=WHEN] \\
               \textit{\textbf{colore la sortie ; WHEN peut être 'always' (par défaut si omis),
              auto', ou 'jamais' ; plus d'informations ci-dessous}}\\

       \par-d, --directory\\
              \textit{\textbf{liste les répertoires eux-mêmes, pas leur contenu}}\\

      \par -D, --dired\\
              \textit{\textbf{génère une sortie conçue pour le mode dired d'Emacs}}\\

       \par-f \\
		 \textit{\textbf{ne pas trier, activer -aU, désactiver -ls --color}}\\

       \par -F, --classify\\
               \textit{\textbf{ajouter un indicateur (l'un des */=>@|) aux entrées}}\\

      \par --file-type\\
              \textit{\textbf{de même, sauf qu'il ne faut pas ajouter '*'.}}\\

       \par --format=WORD\\
              \textit{\textbf{across -x, commas -m, horizontal -x, long -l, single-column -1,
              verbeux -l, vertical -C}}\\

       \par --full-time\\
              \textit{\textbf{comme -l --time-style=full-iso}}\\

       \par-g comme -l\\
	 \textit{\textbf{mais sans liste de propriétaires}}\\

       \par --group-directories-first\\
              \textit{\textbf{regroupe les répertoires avant les fichiers ;
		peut être complété par une option --sort, mais toute utilisation de
              --sort=none (-U) désactive le regroupement.}}\\

       \par -G, --no-group\\
               \textit{\textbf{dans une longue liste, ne pas imprimer les noms de groupe}}\\

       \par-h, --human-readable\\
               \textit{\textbf{avec -l et -s, imprime les tailles comme 1K 234M 2G etc.}}\\

       \par--si \\
		 \textit{\textbf{de même, mais utiliser des puissances de 1000 et non 1024}}\\

        \par-H, --dereference-command-line\\
              \textit{\textbf{suit les liens symboliques listés sur la ligne de commande}}\\

      \par --dereference-command-line-symlink-to-dir\\
              \textit{\textbf{suit chaque lien symbolique de la ligne de commande}}\\

              
       \par--hide=PATTERN\\
              \textit{\textbf{ne pas lister les entrées implicites correspondant au PATTERN de l'interpréteur de commandes (remplacé
              par -a ou -A)}}\\



\end{document}